% HEADER
\documentclass[class=article, crop=false]{standalone}
\usepackage{00_Preamble/frr_preamble}

% Packages
\usepackage{titlesec}
\usepackage{hyperref}
\usepackage{float}
\usepackage{graphics}
\usepackage{placeins}
\usepackage{adjustbox}
% END HEADER

\begin{document}
	\subsection{Competition Purpose}
	\label{subsec:competition_purpose}
	
	
As the Earth’s population increases and its natural resources deplete, it is essential that the world looks to the solar system for its surplus of natural resources. The aerospace industry has made huge strides towards interplanetary travel, and Mars colonization missions are now on the horizon. It is not feasible to frequently supply natural resources from Earth due to launch costs and the lead time on interplanetary travel. With the recent discovery of hydrated minerals in the Martian regolith, it is clear that In-Situ Resource Utilization (ISRU) would be essential to the development of a sustainable Mars civilization. Therefore, it is necessary to develop autonomous robotic systems for efficient water collection on Mars. The NASA Robotic Mining Competition (RMC) was established to promote the development of Mars excavation rovers and inspire the future generation of engineers to pursue ISRU technologies. Each team is tasked with designing, building, and testing a robotic excavation system in a simulated Martian environment. 
	
	
\end{document}
