% HEADER
\documentclass[class=article, crop=false]{standalone}
\usepackage{00_Preamble/frr_preamble}

% Packages
\usepackage{titlesec}
\usepackage{hyperref}
\usepackage{float}
\usepackage{graphics}
\usepackage{placeins}
\usepackage{adjustbox}
% END HEADER

\begin{document}
	\subsection{Purpose of Systems Engineering}
	\label{subsec:systems_engineering_purpose}
	The Vanderbilt Robotics Team followed a systems engineering design process outlined by the V-chart presented below. The excavation robot has many conflicting constraints and multiple subsystems that all need to perform equally well in order to accomplish the presented challenge. The systems engineering process provides a structured methodology to ensure that all system requirements are being satisfied and that all decisions are made to optimize the  and the entire systems is being optimized, instead of specific subcomponents. It provides a holistic, big-picture approach to  the decision making process. 

	
\end{document}
