% HEADER
\documentclass[class=article, crop=false]{standalone}
\usepackage{00_Preamble/frr_preamble}

% Packages
\usepackage{titlesec}
\usepackage{hyperref}
\usepackage{float}
\usepackage{graphics}
\usepackage{placeins}
\usepackage{adjustbox}
% END HEADER

\begin{document}
	\subsection{Problem Statement}
	\label{subsec:problem_statement}
	The robot must be capable of mining and depositing a minimum of 1 kg of icy simulant in a collector bin within 10 minutes. The mining area consists of a 7.38 meter by 3.88 meter container with three zones: the starting zone, obstacle zone, and mining zone. The robot begins the competition run in the starting zone in an unknown orientation. It must traverse the obstacle zone, mine gravel from the mining zone, and return it to the collector bin in the starting zone. The performance of the robot is measured using a point system. Points are awarded and deducted based on the amount of collected gravel as well as design parameters such as the mass of the robot, energy consumption, communication bandwidth utilization, and dust-free operation.

	
\end{document}
