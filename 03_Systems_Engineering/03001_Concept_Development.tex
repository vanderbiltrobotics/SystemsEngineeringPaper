% HEADER
\documentclass[class=article, crop=false]{standalone}
\usepackage{00_Preamble/frr_preamble}

% Packages
\usepackage{titlesec}
\usepackage{hyperref}
\usepackage{float}
\usepackage{graphics}
\usepackage{placeins}
\usepackage{adjustbox}
% END HEADER

\begin{document}
	\subsection{Concept Development}
	\label{subsec:concept_development}
	\subsubsection{Design Philosophy}
	Since this is the team’s first year of competing in the RMC, the design for the robot is entirely new. The team identified three leading design choices that guided the decision making process: maximizing use of off-the-shelf components, optimizing all systems for fully autonomous control, and maximizing the amount of gravel collection. By reducing the number of custom-manufactured components, the lead time on components is reduced, thereby accelerating the assembly timeline. Additionally, off-the-shelf components help to mitigate timeline risk as they reduce the chance that deadlines will be missed due to errors during the fabrication process. Subsystems on the robot were designed for autonomous control under the premise that a fully autonomous mining system would be more valuable for a manned mission. 

	
\end{document}
