%\documentclass{article}}
%
%\begin{document}
\documentclass[class=article, crop=false]{standalone}
\usepackage{00_Preamble/frr_preamble}
\usepackage{graphicx}
\usepackage[table,xcdraw]{xcolor}
\usepackage[normalem]{ulem}
\usepackage{longtable}
\usepackage{hyperref}
\usepackage{float}
\usepackage{graphics}
\usepackage{adjustbox}
\usepackage{placeins}

\begin{document}
\subsubsection{Propulsion Failure Modes}
\begin{footnotesize}
\begin{longtable}{|L{0.12\linewidth}|L{0.12\linewidth}|L{0.11\linewidth}|C{0.05\linewidth}|L{0.22\linewidth}|L{0.16\linewidth}|C{0.08\linewidth}|}

	\caption{Propulsion Failure Modes}\label{table:propulsion_failure_modes}\\ \hline
	\rowcolor[rgb]{ 1,  1,  1} \textbf{Risk} & \textbf{Cause} & \textbf{Effect} & \textbf{RR} & \textbf{Mitigation Strategy} & \textbf{Verification of Mitigation} & \textbf{PMRR} \\
	\hline
	\endfirsthead
	\caption*{Propulsion Failure Modes}\\ \hline
	\rowcolor[rgb]{ 1,  1,  1} \textbf{Risk} & \textbf{Cause} & \textbf{Effect} & \textbf{RR} & \textbf{Mitigation Strategy} & \textbf{Verification of Mitigation} & \textbf{PMRR} \\
	\hline
	\endhead
	Propellant fails to ignite & Improper motor packing; faulty propellant grain; damage during transportation & ``Live" situation; rocket does not launch; necessary replacement & \cellcolor[rgb]{ 1,  0,  0} E3 & Proper ignition setup; safety advisor oversees motor packing by student safety officer. & Consultation of strict safety protocol regarding motor and propellant issues & \cellcolor[rgb]{ .6,  .8,  0} D1 \\
	\hline
	Premature propellant burnout & Improper motor packing; faulty propellant grain & Altitude estimate not reached; main parachute may not deploy & \cellcolor[rgb]{ 1,  0,  0} E3 & Proper motor assembly; obtain motor only from reputable source. & Static fire testing; consultation of safety protocol & \cellcolor[rgb]{ 1,  1,  0} E1 \\
	\hline
	Improper assembly of motor & Incorrect spacing between propellant grains; motor case improperly cleaned; end caps improperly secured & Motor failure; unstable flight; target altitude not reached; damage or loss of rocket & \cellcolor[rgb]{ 1,  0,  0} E3 & Ensure proper training and supervision by safety advisor for motor assembly by student safety officer. & Consultation of strict safety protocol regarding motor and propellant issues & \cellcolor[rgb]{ 1,  1,  0} E1 \\
	\hline
	Motor mount fails & Insufficient mount strength; damage during previous launch or transportation & Motor launches through rocket; damage to/loss of rocket; unstable flight & \cellcolor[rgb]{ 1,  0,  0} E3 & Proper motor mount construction; load verification testing; test launches. & Load verification testing; design analysis of motor mount; pre- and post-flight inspections of motor mount \ref{subsec:motor_installation_and_cg_operations} & \cellcolor[rgb]{ 1,  1,  0} E1 \\
	\hline
	Transportation/ handling damage & Improper protection during transportation/ handling & Unusable motor; incapable of safe launch; potential damage to/loss of rocket & \cellcolor[rgb]{ 1,  0,  0} E3 & Proper storage overseen by safety advisor and student safety officer; certified member handling. & Consultation of strict safety protocol regarding motor and propellant issues & \cellcolor[rgb]{ 1,  1,  0} E2 \\
	\hline
	Center ring failure & Unable to withstand motor force during launch; weak ring; poor seal to body and motor tube & Reduced stability; damage to/loss of vehicle & \cellcolor[rgb]{ 1,  0,  0} E3 & Proper ring size and construction; sufficiently strong materials used (6061-T6 aluminum); redundant load path design that can sustain failure of fins or centering rings and still retain motor. & Finite element modeling to verify rings to hold conservative thrust loads using von Mises failure criterion; See CDR for FEA & \cellcolor[rgb]{ 1,  1,  0} E2 \\
	\hline
	Propellant explodes & Improper motor packing; faulty propellant grain; damage during transportation & Destruction of motor casing; catastrophic failure of rocket; potential injury to personnel & \cellcolor[rgb]{ 1,  1,  0} E2 & Proper motor assembly; safety advisor oversees motor packing by student safety officer. & Consultation of strict safety protocol regarding motor and propellant issues & \cellcolor[rgb]{ 1,  1,  0} E1 \\
	\hline
	Propellant burns through casing & Improper motor packing; faulty propellant grain; damage during transportation & Loss of thrust; loss of stability; catastrophic failure of rocket & \cellcolor[rgb]{ 1,  1,  0} E2 & Proper motor assembly; safety advisor oversees motor packing by student safety officer. & Static fire testing to verify proper motor assembly
	& \cellcolor[rgb]{ 1,  1,  0} E1 \\
	\hline
	Motor tube dislodges from rocket body during launch & Failure of fin attachment, exposing motor tube connection
	& Catastrophic launch failure, uncontrolled flight & \cellcolor[rgb]{ 1,  1,  0} E2 & Thorough construction of motor tube mounting through fins. For the motor tube to tear out, the fins would have to tear through the carbon fiber body. & Design analysis of tail section structure; visual inspection of tail section pre-flight \ref{subsec:tail_section} & \cellcolor[rgb]{ 1,  1,  0} E1 \\
	\hline
	Motor is misaligned & Centering rings misaligned; fins assembled to motor tube at an angle & Unexpected flight trajectory; unstable flight & \cellcolor[rgb]{ 1,  1,  0} C4 & Careful machining of center rings on lathe with order of magnitude higher tolerance than laser cut plywood; proper assembly of tail section using centering rings and fin alignment jig. & Design analysis of motor alignment equipment; pre-flight visual inspection of motor alignment \ref{subsec:motor_installation_and_cg_operations} & \cellcolor[rgb]{ .6,  .8,  0} C1 \\
	\hline
	Motor igniter fails & Faulty or incorrect igniter & ``Live" situation; rocket does not launch; necessary replacement & \cellcolor[rgb]{ 1,  1,  0} C2 & Proper igniter selection setup; proper power source. & Adherence to safety protocol
	& \cellcolor[rgb]{ .6,  .8,  0} C1 \\
	\hline
\end{longtable}
\end{footnotesize}
\end{document}
